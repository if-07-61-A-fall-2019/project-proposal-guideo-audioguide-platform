\documentclass[12pt]{article}
\usepackage{geometry}                % See geometry.pdf to learn the layout options. There are lots.
\geometry{letterpaper}                   % ... or a4paper or a5paper or ...
\usepackage{graphicx}
\usepackage{amssymb}
\usepackage{amsthm}
\usepackage{epstopdf}
\usepackage[utf8]{inputenc}
\usepackage[usenames,dvipsnames]{color}
\usepackage[table]{xcolor}
\usepackage{hyperref}
\usepackage{parskip}
\DeclareGraphicsRule{.tif}{png}{.png}{`convert #1 `dirname #1`/`basename #1 .tif`.png}
 
\theoremstyle{definition}
\newtheorem{example}{Example}

\newenvironment{explanation}{%
   \setlength{\parindent}{0pt}
   \itshape
   \color{blue}
}{}

\newenvironment{text}{
}{} 
 
\newcommand{\productname}{Guideo - Audio Guide Platform}
\newcommand{\projectleader}{L. Engleder}
%\newcommand{\documentstatus}{  In process }
\newcommand{\documentstatus}{Submitted}
%\newcommand{\documentstatus}{Released}
\newcommand{\version}{V. 1.7.0}
 
\begin{document}
\begin{titlepage}
\begin{flushright}
\includegraphics[scale=.5]{htlleondinglogo.png}\\
\end{flushright}
 
\vspace{10em}
 
\begin{center}
{\Huge Project Proposal} \\[3em]
{\LARGE \productname} \\[3em]
\end{center}
 
\begin{flushleft}
\begin{tabular}{|l|l|}
\hline
Project Name & \productname \\ \hline
Project Leader & \projectleader \\ \hline
Document state & \documentstatus \\ \hline
Version & \version \\ \hline
\end{tabular}
\end{flushleft}
 
\end{titlepage}
\section*{Revisions}
\begin{tabular}{|l|l|l|}
\hline
\cellcolor[gray]{0.5}\textcolor{white}{Date} & \cellcolor[gray]{0.5}\textcolor{white}{Author} & \cellcolor[gray]{0.5}\textcolor{white}{Change} \\ \hline
September 16, 2019&L. Engleder/P. Quoc/L. Wirth&First version \\ \hline
September 20, 2019&L. Engleder/P. Quoc/L. Wirth& Base information for each section; \\ && proofreading by P. Bauer \\ \hline
September 23, 2019&L. Engleder/P. Quoc/L. Wirth&Minor improvements \\ && and more details;  \\ && proofreading by P. Bauer \\ \hline
September 27, 2019&L. Engleder/P. Quoc/L. Wirth/A. Leeb&writing on planing section \\ \hline
September 29, 2019&L. Engleder/P. Quoc/L. Wirth/A. Leeb&more information in \\ && planning section \\ \hline
September 30, 2019&L. Engleder/P. Quoc/L. Wirth/A. Leeb& improvements on all sections \\ \hline
October 4, 2019&L. Engleder/P. Quoc/L. Wirth/A. Leeb&improvements on all sections\\ \hline
October 7, 2019&L. Engleder/P. Quoc/L. Wirth/A. Leeb&final version\\ \hline
\end{tabular}
\pagebreak
 
\tableofcontents
\pagebreak
 
\section{Introduction}
\begin{text}
Guideo is a platform for users and creators of audio guides alike. Depending on the location the user can choose between a variety of guides, which in turn deliver
an informative and highly interesting experience. Someone could for example travel around the city without any type of knowledge or guide and learn about a variety of subjects
such as culture and history. Our app would also enable a simple way of adding guides to the catalogue.   
 
\end{text}
\pagebreak
 
\section{Initial Situation}
\begin{text}
Currently there are some ways for people to get information about their travelling destination.\newline

An established but also somewhat outdated way is to just read about it in a guide book or online.
Buying a generic guide book and
reading page to page is a tedious task for most people. It takes hours to get into the often dry material and often results in holey knowledge of the city.\newline

And the problem with online travel websites is the huge amount of unfiltered information which distracts from the really interesting things. Moreover there is no way to really enjoy these things in real-time and location based.\newline
 
Another option for many people are tour guides and bus tours which promise to offer an educative personalized experience. %In the short description not included are their high prices 
When you consider their high prices many of these tours still happen to have a lot of inconveniences and problems. First of all, there is a huge amount of people who are unable to take part in tours because of hearing impairment or a slower walking speed. An important thing to note is also the small pool of languages these services typically offer. Moreover it can simply be an awful experience to walk around in an awkward group of strangers who you have to adjust to.\newline
 
For many museums the installment of an audio guide system can be a big financial burden. Most of all the purchase of old outdated hardware for high prices from specialized companies seems out of place in our modern connected world. Keeping enough of these things working probably requires a separate technician for maintenance.
Moreover without regular inspection and cleaning, these systems fail basic hygienic standards. \newline

Although there are archaic audio guides in use in museums there is still no way to enjoy them in a city in general. Our service would provide both locations with fitting guides.
 
\end{text}
 
\pagebreak
 
\section{General Conditions and Constraints}

\subsection{Framework conditions} 
\begin{text}

We are not planning to spend any money except for the cost of a web hosting server. Additional budget are possible but not really wanted.\newline

As the app was initially and is still imagined as a mobile app. We would develop the app with Angular, a front-end framework, capable of delivering cross-platform mobile solutions. For the server functionality we will use Java or NodeJS .\newline
 
At the time of writing only one of our members gathered experiences with Angular, but we are expected to be taught Angular in the next months in school.\newline
 
We wouldn't cooperate with any other project in our class but if there are museums showing interest in our project we would try to integrate their audio guides
which are already in use. It would greatly expand our catalogue of guides and help us establish ourselves in the world of museums.\newline
 
There are no commitments as of now. But we plan to contact some museums and galleries in the near future.

\end{text}

\subsection{Technical conditions}
\begin{text}

We will develop the app with Angular and the server functionality in combination with IntelliJ.

Versioning will be managed over GitHub.\newline
 
There are currently only our 4 Laptops in use and no Server structure available. And 4 android mobile devices for testing.\newline
 
We will use English when documenting and programming.

\end{text}
  
\pagebreak
 
\section{Project Objectives and System Concepts}
 
\subsection{General}
\begin{text}
In general, we aim to create the same experience you could get in museums with their audio guides, but easily accessible through our app. Normally these guides are only available in museums but our routes will be available at iconic sights, in cities, zoos and much more. All of this at fingertips' reach through our mobile application. 

Further on, the museum and zoos in question will easily be able to provide audio guides to their visitors, completely eliminating the need for expensive equipment and maintenance staff.

They only have to provide the audio files and each location on the map or in the building. Our app will recognize if the user is near a placed audio file and play the desired track.

Not only museums or other institutions can create guides, but every one who is willing to create their own experience is welcomed to upload it to our platform for other users to enjoy. These users can then choose between a variety of audio guides covering different topics and presenting the information in various styles.
\end{text}
 
\subsection{Features}
\begin{text}
The users can listen to audio guides, if they are near a specific location. These audio guides will be provided by verified users (e. g. museums, cities, etc.) or the community. Users with popularity among the community and a clean record of guides will be given the option to upgrade to a verified user.
When provided and used in a museum the Geo-location based system would be replaced with a tagging system more suited to such an environment.
We are planning on offering different options for museums to install their tagging system. The most modern but also most expensive solution would be installment of beacons that locate users in small spaces and make the experience fluid and easy.
Another, more simple, solution would be to tag the stations or artworks with simple numbers combined with QR-codes which are easily scanned and integrated into the experience.\newline
 
To clarify, a guide is a list of different tracks/files each telling a story or giving information about a certain location/object. If a user decides to create a guide he will be given the option to fully create all of its tracks or to build a compilation out of other guides and his own files. A guide should be a creative medium encompassing all the stylistic decision the guide creator made. \newline

But if a listener encounters a location or object he wants to hear about he will be notified and provided with an explanation of another guide. Still, it will not be played automatically, as we think that if a creator decides not to include a location it was a choice he made out of good reasons.
Out of concern that users could miss important locations or that the amount of points in one guide is low we will offer a Top Track Mode. Which automatically plays the most popular track of a specific location.\newline
 
Users that provide audio guides can get verified. Only verified audio guides must be within the regulations and rules. Non-verified audio guides won’t be checked but they need to respect the rules. Users who intentionally provide misinformation, advertise products, etc. will be banned.\newline
 
Users can leave a rating on the audio guides. Inappropriate, faulty or wrong audio guides can be reported by the community. The management of guides will be readily available on e app and a separate website which provides the needed features for uploading, updating and management in general.\newline

To provide people with recommendations and an optimal user experience we plan on collecting some essential data like the languages he or she speaks, preferred language, favorite topics and obviously basic information such as name and email.\newline
 
A notification will pop up if the user is close to a place or point where a verified audio guide was set, but only if the application is running. A pop up window will appear if the app runs in the foreground. Otherwise, the notification will show up on the status bar and in the notification inbox. But only for verified audio guides will a notification popped up. Non-verified audio guides will only be shown in the Explorer View.\newline
 
In the Explorer View the user can explore all available audio guides (verified and non-verified) on a google maps like map. Alternatively there will be a list of audio guides structured around locations (e. g. Linz - Linz Kernzone) with essential information such as language, creator, length of time, topics and popularity. Filtering and Sorting will be enabled.
\end{text}

\subsection{Financial}
\begin{text} 
Our aim is that most of the audio guides are cost-free and that our revenue stems from the purchase of audio guides and advertisements.\newline

To financially sustain ourselves we will use a combination of different sources of revenue. We will employ advertisement on free guides at the beginning and end of the tour. This feature will also push users to buy the ad-free application in exchange for money but, this does not mean that all audio guides (like museums, zoos...) on our platform will be provided for free. When selling the guides we will also take a small commission for our services.\newline

Our platform should also provide a payment system to make the purchase of specific audio guides (museums, zoos...) easy and intuitive. 
\end{text}

\subsection{Testing}
\begin{text}
We are planning to test the tagging system in our own school as it shares some features as size and basic architectural structure with a normal museum. It could even be used in events such as ToT to showcase our application and obviously test our systems under more realistic conditions.\newline
\end{text}
 
\pagebreak
\section{Opportunities and Risks}
\begin{text}
The project could acquire popularity with many people who are visiting another country and want an easy way to learn about customs, culture and history.
Also cities might be interested in this project in order to increase tourism due to the popularity of our audio guide catalogue.
It could also turn out successful as a creative platform for creators of this somewhat new medium. We are also exploring the possibility of selling these created guides. The user will be compensated fairly.\newline
 
Another opportunity of expanding the app would be to simulate a audio-visual experience through the use of AR-technology. It could be employed in historical settings and sceneries. Although the idea is very interesting it would require far more development time and expertise in the field.\newline

Opportunities also lie in our current school as it could be used in events such as ToT to showcase our application and obviously test our systems under more realistic conditions.
 
We could face the risk that our catalogue is too small to really consider it as an option.\newline 
 
One of the biggest risk would just be the absence of any traction and popularity among Museums, Zoos and the like. \newline
 
\end{text}
 
\pagebreak
\section{Planning}

\subsection{Milestones}
\begin{itemize}
\item Database Structure and Development
\item Localization and Notifications
\item Audio Guide Streaming
\item Route Finding
\item Account Login, Authentication and Management
\item Tagging System
\item Payment System
\item Guide Creation System (Recording and Tagging)
\item Archive + Rating + Reporting
\end{itemize}

\subsection{Members}
\begin{tabular}{|l|l|}
\hline
\cellcolor[gray]{0.5}\textcolor{white}{Name} & \cellcolor[gray]{0.5}\textcolor{white}{Role}\\ \hline
Lucas Engleder & Project Leader, Mobile Co-Master and Database Assistant\\ \hline
Patrick Quoc & Mobile Master\\ \hline
Lukas Wirth & Seamodea \\  \hline
Alexander Leeb & Database Master and Server Assistant \\ \hline
\end{tabular}

Role Explanation:
\begin{itemize}
    \item Seamodea
    \begin{itemize}
        \item \textbf{Se}rver
        \item \textbf{A}nd
        \item \textbf{Mo}bile \textbf{De}sign
        \item \textbf{A}rchitect
    \end{itemize} 
\end{itemize}
\subsection{Resources}
\subsubsection{Human Resources}
\begin{text}
Our project is actually a school project which disables to really consider external programming help outside of our core team members. We wouldn't use this option anyway as we see this project not only as a project but as an opportunity to learn new things and try innovative technologies.

But regarding design, we are bearing in mind that our capabilities might be limited and in need of an overlooking eye. Possible candidates are colleagues in our school's design branch.\end{text}

\subsubsection{Licenses and Server}
\begin{itemize}
\item License for Jetbrains' IntelliJ Ultimate Edition IDEA to develop Angular Apps.
\item Microsoft's Visual Studio Code and Drifty Co.'s Ionic are under the MIT License.
\item Database License
\end{itemize}

We are considering buying a server if our school isn't able to provide one. In the case of a possible purchase we would need a server with storage capabilities to accommodate our file space needs.

\subsection{Project Management}
\begin{itemize}
    \item Start of project: 30th September 2019
    \item End of project: End of 5th grade
    \newline
    \item First Prototype available: 24.02.2020 (First day of school after semester vacation)
    \item Begin of implementation work: 11th October 2019
    \newline
    \newline
    \item Big blocks of work
    \begin{itemize}
        \item Database and Server Structure
        \item Localization and Notifications
        \item Audio Guide Streaming
        \item Account Login, Authentication and Management
        \item Route Finding and Google maps integration in general
    \end{itemize}
    \item With enough dedication and without major problems in development we estimate it to be hard but quite possible
    \item As already mentioned we will need a server for our application.
    
\end{itemize} 

\end{document}  
